\documentclass{article} 
\usepackage{graphicx}
\usepackage{float}
\begin{document}
\begin{titlepage}
	\begin{figure}
		\centering
		\includegraphics[height=1.5in]{muk.jpg}
	\end{figure}
	\begin{center}
		\line(1,0){320}\\
		[0.25in]
		\huge{\bfseries BUILDING MATHEMATICAL OBJECTS DATABASES USING SEGMATH}\\
		[2mm]
		\line(1,0){150}\\
		[1.2cm]
		\textsc{\large COLLEGE OF COMPUTING AND INFORMATICS TECHNOLOGY}\\
		[0.5cm]
		\textsc{\large By}\\
		[0.5cm]
		\textsc{\large Group 204}\\
		[4cm]
	\end{center}
	\begin{flushright}
	    Research\\
		Methodology\\
		cls\\
	\end{flushright}
\end{titlepage}
\thispagestyle{empty}
\centering
	\textsc{\large GROUP MEMBERS} \\
	\line(1,0){115}\\
	[0.25in]
\begin{table}[H]
	\centering
	\label{Tab:GroupMembers}
	\begin{tabular}{lr}
		\bfseries{Name} & \bfseries{Registration Number} \\ \\ 
		OKOTH JAMES         & 15/U/20773/EVE \\
		FAHAD HASSAN        & 13/U/5186/EVE \\
		BUSUULWA CHARLES    & 14/U/5870/PS \\   
		KANYESIGYE EMMANUEL & 15/U/21140/EVE \\
	\end{tabular}
\end{table}
\newpage
\thispagestyle{empty}
\cleardoublepage
\setcounter{page}{1}

\section{INTRODUCTION}\label{sec:intro}
We have been studying about a lot of staff in computer science here at Makerere for years now, but we have never come up with any practically working model of any application of what we have covered so far. This project intends to discover the uncharted mathematical worlds. \\

\section{BACKGROUND}
For a while there has been no computerized online collection of mathematics objects for students to browse and just like explorers, mathematicians seek to discover paths between apparently unrelated areas. Such discoveries can lead to breakthroughs when the connections are made explicit. For example, in the 17th century Rene Descartes forged a revolutionary connection between geometry and algebra. 
There is an enormous amount of information on constructing various sorts of “interesting”, in one or another way, mathematical objects, e.g. block designs, linear and non-linear codes, Hadamard matrices, elliptic curves, etc.  There is considerable interest in having this information available in computer-ready form.  However, usually the only available form is a paper describing the construction, while no computer code and often no detailed description of a possible implementation is provided. This provides interesting algorithmic and software engineering challenges in creating verifiable implementations; properly structured and documented code, supplemented by unit tests, has to be provided, preferably in functional programming style. \\

\section{PROBLEM STATEMENT}
Due to an increased large-scale cloud computing which is one of the ways to provide sophisticated web interfaces that allow both experts and amateur to easily navigate their contents, there is a problem of uncharted mathematical terrain which requires online resources that provides detailed maps for mathematics. This can be solved by building mathematical objects databases using segmath for charting the terrain of rich, new mathematical worlds, and sharing of discoveries of the best mathematicians over the web.   \\

\section{OBJECTIVES}

\subsection{MAIN OBJECTIVE}
To discover uncharted mathematical terrain that provides detailed maps for mathematics in computers both locally and remotely by building mathematical objects databases using segmath.\\

\subsection{SPECIFIC OBJECTIVES}
To let best mathematicians all over the world share knowledge in the mathematics world through online resources.\\ 
To give students access to mathematics objects through computers and other network devices available.\\
To allow researchers do extensive research about the topic and the project in general.\\
\cleardoublepage
\bibliographystyle{IEEEtran}
\section{REFERENCES}\label{sec:intro}
{{https://www.cs.ox.ac.uk/teaching/studentprojects/470.html},
	editor = {Oxiford},
	title = {Math objects},
	date = {1/3/2013},
}
{{http://ucsdnews.ucsd.edu/pressrelease/researchersdevelopnewwaytoexploremathematicaluniverse}		,
	editor = {Oxiford},
	title = {Math objects},
	date = {13/8/2010},
}

{{http://www.mathesia.com/community/the-atlas-of-mathematical-objects/},
	editor = {Oxiford},
	title = {Math objects},
	date = {18/9/2016},
}

{{http://news.mit.edu/2016/international-team-launches-atlas-mathematical-objects-0510},
	editor = {Oxiford},
	title = {Math objects},
	date = {1/03/2015},
}

{{http://www.sciencealert.com/researchers-are-building-a-huge-mind-bending-atlas-of-the-mathematical-universe},
	editor = {Oxiford},
	title = {Math objects},
	date = {10/06/2014},
}

 
\end{document}



