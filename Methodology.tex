\documentclass[11pt]{report}
\title{\textbf{BUILDING DATABASES OF MATHEMATICAL OBJECTS IN SAGEMATH (PYTHON)}}

\author{COLLEGE OF COMPUTING AND INFORMATION SCIENCES\\DEPARTMENT OF COMPUTER SCIENCE\\RESEARCH METHODOLOGY}

\begin{document}
\maketitle

\section{INTRODUCTION}
A Mathematical Objects Database can be like a museum with all of best mathematical specimens is an intricate catalog and the connections between them. SageMath is a free open-source mathematics software system licensed under the General Public License. It builds on top of many existing open-source accessing a combined power through a common Python based language.

\section{BACKGROUND ABOUT THE PROBLEM}

\section{PROBLEM STATEMENT}
Due to an increased large-scale cloud computing which is one of the ways to
provide sophisticated web interfaces that allow both experts and amateur to
easily navigate their contents, there is a problem of uncharted mathematical
terrain which requires online resources that provides detailed maps for
mathematics.
\section{OBJECTIVES}
\subsection{Main Objective}
To build a Mathematical Objects database that provides detailed maps for
mathematics in computers both locally and remotely.

\subsection{Specific Objectives}
To use the necessary methodology to carry out research and test the database.
To give students access to mathematics objects through computers and other
network devices available.

\section{METHODOLOGY}
We shall get data from sources like text books and implement most of the infinite families of graphs listed there in the open-source software Sagemath, as well as provided constructions of the sporadic cases, to obtain a graph for each set of parameters with known examples.
\section{SCOPE}
\subsection{Geographical Scope}


\subsection{Functional Scope}


\subsection{Durational Scope}


\section{SIGNIFICANCE}


\section{RECCOMENDATION AND CONCLUSION}


\end{document}