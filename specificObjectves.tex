% !TEX TS-program = pdflatex
% !TEX encoding = UTF-8 Unicode

% This is a simple template for a LaTeX document using the "article" class.
% See "book", "report", "letter" for other types of document.

\documentclass[11pt]{article} % use larger type; default would be 10pt

\usepackage[utf8]{inputenc} % set input encoding (not needed with XeLaTeX)

%%% Examples of Article customizations
% These packages are optional, depending whether you want the features they provide.
% See the LaTeX Companion or other references for full information.

%%% PAGE DIMENSIONS
\usepackage{geometry} % to change the page dimensions
\geometry{a4paper} % or letterpaper (US) or a5paper or....
% \geometry{margin=2in} % for example, change the margins to 2 inches all round
% \geometry{landscape} % set up the page for landscape
%   read geometry.pdf for detailed page layout information

\usepackage{graphicx} % support the \includegraphics command and options

% \usepackage[parfill]{parskip} % Activate to begin paragraphs with an empty line rather than an indent

%%% PACKAGES
\usepackage{booktabs} % for much better looking tables
\usepackage{array} % for better arrays (eg matrices) in maths
\usepackage{paralist} % very flexible & customisable lists (eg. enumerate/itemize, etc.)
\usepackage{verbatim} % adds environment for commenting out blocks of text & for better verbatim
\usepackage{subfig} % make it possible to include more than one captioned figure/table in a single float
% These packages are all incorporated in the memoir class to one degree or another...

%%% HEADERS & FOOTERS
\usepackage{fancyhdr} % This should be set AFTER setting up the page geometry
\pagestyle{fancy} % options: empty , plain , fancy
\renewcommand{\headrulewidth}{0pt} % customise the layout...
\lhead{}\chead{}\rhead{}
\lfoot{}\cfoot{\thepage}\rfoot{}

%%% SECTION TITLE APPEARANCE
\usepackage{sectsty}
\allsectionsfont{\sffamily\mdseries\upshape} % (See the fntguide.pdf for font help)
% (This matches ConTeXt defaults)

%%% ToC (table of contents) APPEARANCE
\usepackage[nottoc,notlof,notlot]{tocbibind} % Put the bibliography in the ToC
\usepackage[titles,subfigure]{tocloft} % Alter the style of the Table of Contents
\renewcommand{\cftsecfont}{\rmfamily\mdseries\upshape}
\renewcommand{\cftsecpagefont}{\rmfamily\mdseries\upshape} % No bold!

%%% END Article customizations

%%% The "real" document content comes below...

\title{BUILDING DATABASES FOR MATHEMATICAL OBJECTS USING SAGEMATH}
\author{GROUP 204}
%\date{} % Activate to display a given date or no date (if empty),
         % otherwise the current date is printed 

\begin{document}
\maketitle

\section{INTRODUCTION}
A mathematical objects is an abstract object arising in mathematics.
In mathematical practice, an object is anything that has been (or could be) formally defined, and with which one may do deductive reasoning and mathematical proofs.

Examples inclide: numbers, permutations,partitions, matrices, sets, functions, and relations.
categories such as algebra and geometry are simultaneously homes to mathematical objects and are mathematical objects in their own right.
A Mathematical Objects Database can be like a museum with all of best
mathematical specimens in an intricate catalog and the connections
between them.
SageMath is a free open-source mathematics software
system licensed under the General Public License. It builds on top of
many existing open-source accessing a combined power through a common
Python based language.


\section{PROBLEM STATEMENT}

Due to an increased large-scale cloud computing which is one of the ways to provide sophisticated web interfaces that allow both experts and amateur to easily navigate their contents, there is a problem of uncharted mathematical terrain which requires online resources that provides detailed maps for mathematics.

 This can be solved by building databases of mathematical objects for charting the terrain of rich, new mathematical worlds, and sharing of discoveries of the best mathematicians over the web.


\section{MAIN OBJECTIVE}

To build a database of Mathematical Objects that provides detailed maps for mathematics in computers both locally and remotely, plus a documentation on how application developers can use the database.

\subsection{SPECIFIC OBJECTIVES}
 To collect information regarding the different types of mathematical objects and analysing it.

 To collect information on how to buld databases in python.
 


\section{METHODOLOGY}

Your text goes here.

\subsection{A subsection}

More text.
\section{REFERENCES}

Your text goes here.

\subsection{A subsection}

More text.
\end{document}
